\chapter*{THESIS INFORMATION}
\addcontentsline{toc}{chapter}{{\bf THESIS INFORMATION}}

\begin{flushleft}
Thesis title: A study on Global Error Bounds for Convex Inequalities Systems

Speciality: Applied Mathematics

Code: 84 60 112

Name of Master Student: Le Nhut Nam

Academic year: 33/2023

Supervisor: Prof. Dr. Nguyen Van B

At: VNUHCM - University of Science
\end{flushleft}

\section*{1. SUMMARY}

This thesis investigates the existence and characterization of global error bounds for convex functions in Euclidean spaces, providing necessary and sufficient conditions without imposing additional requirements on the functions or their solution sets. By focusing on Euclidean spaces, we derive conditions that are straightforward to verify, enhancing practical applicability.

Furthermore, we explore the subdifferential properties of the supremum function formed by an arbitrary family of convex continuous functions. Specifically, we demonstrate that, under a suitable assumption, the subdifferential of this supremum function corresponds to the convex hull of the subdifferentials of the individual functions at active indices. This result extends our understanding of the relationship between the subdifferentials of component functions and their supremum, offering a new perspective on convex analysis.

The applications of these theoretical findings are particularly relevant for the study of global error bounds in systems characterized by infinite linear and convex inequalities. Our results not only generalize existing theories but also provide practical criteria that are simpler to apply.

Throughout the thesis, several examples illustrate the advantages of the proposed conditions over existing ones, demonstrating the improved efficacy and utility of our approach in various contexts. The examples underscore the practical significance of our theoretical contributions, highlighting their potential impact on solving complex optimization problems involving convex functions.

\section*{2. NOVELTY OF THESIS}

This thesis introduces a novel approach by developing the theory of subdifferential and singular subdifferential without requiring any assumptions of closedness and lower semicontinuity on the sets and functions. This marks a significant advancement compared to previous research, where such assumptions were typically considered necessary. By eliminating these stringent requirements, the thesis offers more flexible approaches for the analysis and application of convex functions. As a result, the developed theory is not only easier to verify and apply but also extends the range of optimization problems that can be addressed, particularly in the context of systems with infinite linear and convex inequalities.

\section*{APPLICATIONS/ APPLICABILITY/ PERSPECTIVE}

The applications of this thesis are particularly impactful in the study of optimization problems within Banach spaces, focusing on the Pshenichnyi-Levin-Valadier (PLV) property. By establishing necessary and sufficient conditions for the PLV property without the need for assumptions of closedness and lower semicontinuity, the thesis provides a robust framework for analyzing and solving complex optimization problems in these spaces. This contributes to a deeper understanding of the structural properties of convex functions and inequalities in Banach spaces, enhancing the ability to address a wider range of practical problems. The findings have significant implications for fields requiring sophisticated optimization techniques, such as economics, engineering, and applied mathematics, where the PLV property plays a critical role in modeling and solving real-world problems.

\vspace{4\baselineskip}
\begin{table}[H]
\begin{adjustbox}{max width =\textwidth}
\begin{tabular}{p{8.44cm}p{8.4cm}}
\multicolumn{1}{p{8.44cm}}{
\centering \textbf{SUPERVISOR} \newline
\centering
(Ký tên, họ tên) \newline
} &
\multicolumn{1}{p{8.4cm}}{
\centering \textbf{Master STUDENT} \newline
\centering
(Ký tên, họ tên) \newline
} \\
\end{tabular}
\end{adjustbox}
\end{table}
\vspace{2\baselineskip}
\begin{center}
    \textbf{CERTIFICATION \\ UNIVERSITY OF SCIENCE}
\end{center}
\begin{center}
    \textbf{PRESIDENT}
\end{center}