\chapter{GIỚI THIỆU}

Kết quả tiên phong của Hoffman về chặn sai số cho các hệ bất đẳng thức affine trong không gian hữu hạn chiều đã đóng một vai trò quan trọng trong các bài toán quy hoạch toán học. Nghiên cứu về sự tồn tại của các chặn sai số dẫn đến nhiều ứng dụng trong nhiều lĩnh vực như phân tích độ nhạy, tính toán đạo hàm dưới và sự hội tụ của các phương pháp số, v.v. Hướng phát triển chính của các kết quả nổi tiếng của Hoffman cho đến nay là mở rộng chúng sang các hệ thống khác nhau trong các không gian vô hạn chiều. Robinson là một nhân vật quan trọng trong quá trình này từ đầu. Trong một công trình, ông đã chỉ ra rằng các chặn sai số toàn cục cho một hàm lồi liên tục tồn tại khi tập nghiệm bị chặn và điều kiện Slater được thỏa mãn. Ioffe đã nghiên cứu các chặn sai số (dưới một tên khác) dưới giả định rằng các hàm số là Lipschitz cục bộ trên các không gian Banach. Kể từ những công trình cơ bản này, rất nhiều đóng góp cho lý thuyết về các chặn sai số đã được thực hiện bởi nhiều tác giả. Nhìn chung, hầu hết các tiêu chí chặn sai số đều được trình bày cho các hàm liên tục, nửa liên tục dưới hoặc các ràng buộc trên tập nghiệm.

Nhớ lại rằng một tập con của không gian Banach được gọi là đều lồi nếu nó là giao của một họ tùy ý, có thể rỗng, các nửa không gian mở, và một hàm từ không gian Banach tới các số thực mở rộng được gọi là đều lồi nếu tập mở rộng của nó là một tập đều lồi. Đều lồi ban đầu được giới thiệu trong trường hợp hữu hạn chiều bởi Fenchel và từ đó trở thành một khái niệm đáng chú ý trong phân tích lồi. Đặc biệt, các bài toán tối ưu hóa đều lồi có nhiều ứng dụng và đã là đối tượng của nghiên cứu chuyên sâu trong vài thập kỷ qua. Cuốn chuyên khảo gần đây "Even Convexity and Optimization" của Fajardo và các cộng sự là một tài liệu tham khảo xuất sắc cho những ai quan tâm đến chủ đề này. Như đã chỉ ra trong một số công trình gần đây, các hàm lồi nửa liên tục dưới là các trường hợp đặc biệt của các hàm đều lồi, trong khi bất kỳ hàm đều lồi nào cũng là hàm lồi. Điều này có nghĩa là lớp các hàm lồi rộng hơn đáng kể so với lớp các hàm lồi nửa liên tục dưới. Do đó, lý thuyết về các chặn sai số cho các hàm lồi mà không giả định tính nửa liên tục dưới là một chủ đề đáng để nghiên cứu.

Mặt khác, khái niệm về điều kiện đủ ràng buộc cho các hệ bất đẳng thức lồi hữu hạn cũng đóng một vai trò quan trọng trong tối ưu hóa. Nhiều mở rộng của khái niệm này đã được nghiên cứu trong tài liệu bằng cách tập trung vào các hệ vô hạn trong không gian vô hạn chiều. Lưu ý rằng nhiều tiêu chí cần và đủ cho sự tồn tại của các chặn sai số đã được trình bày thông qua các điều kiện đủ ràng buộc. Ví dụ, Zheng và Ng đã thiết lập một đặc trưng của các chặn sai số cục bộ cho các bất đẳng thức lồi dưới dạng điều kiện đủ ràng buộc cơ bản mạnh mẽ; sau đó Hu đã có một cải tiến đáng kể; Ngai đã chứng minh sự tương đương giữa chặn sai số toàn cục kiểu Lipschitz cho các hệ bất đẳng thức đa thức lồi hữu hạn và điều kiện đủ ràng buộc Abadie; Boţ và Csetnek đã cung cấp một tiêu chí đủ cho sự tồn tại của các chặn sai số toàn cục dưới dạng sự bị chặn của tập nghiệm và điều kiện đủ ràng buộc Slater, v.v. Gần đây, Chuong và Jeyakumar đã trình bày một số đặc trưng thú vị của các chặn sai số vững chắc cho các hệ bất đẳng thức tuyến tính dưới sự không chắc chắn dữ liệu thông qua các điều kiện liên quan đến điều kiện đủ ràng buộc cơ bản. Là một ứng dụng, họ đã suy ra sự tồn tại của các chặn sai số vững chắc trong các trường hợp không chắc chắn kịch bản thông thường và không chắc chắn khoảng. Ngoài ra, cần nhấn mạnh ở đây rằng các điều kiện đủ ràng buộc cho các hệ vô hạn có liên quan chặt chẽ đến cái gọi là tính chất Pshenichnyi-Levin-Valadier (gọi tắt là PLV).

Được truyền cảm hứng từ các quan sát trên, chúng tôi tinh chỉnh các kết quả gần đây theo hai hướng. Đầu tiên, chúng tôi nhận thấy rằng tính nửa liên tục dưới của một hàm lồi và các ràng buộc trên tập nghiệm trong lý thuyết các chặn sai số toàn cục có thể được bỏ qua. Thứ hai, chúng tôi mở rộng một kết quả (tính chất PLV) đã có từ các không gian hữu hạn chiều sang các không gian Banach tổng quát. Là các ứng dụng, chúng tôi trình bày các đặc trưng của các chặn sai số toàn cục cho các họ tùy ý của các hàm tuyến tính và lồi.