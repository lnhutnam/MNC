\chapter*{TRANG THÔNG TIN LUẬN VĂN}
\addcontentsline{toc}{chapter}{{\bf TRANG THÔNG TIN LUẬN VĂN}}

\begin{flushleft}
Tên đề tài luận văn: Nghiên cứu về chặn sai số toàn cục cho hệ bất phương trình lồi

Ngành: Toán Ứng Dụng

Mã số ngành: 84 60 112

Họ tên học viên cao học: Lê Nhựt Nam

Khóa đào tạo: 33/2023

Người hướng dẫn khoa học: GS. TS. Nguyễn Văn B

Cơ sở đào tạo: Trường Đại học Khoa học Tự nhiên, ĐHQG.HCM 
\end{flushleft}

\section*{1. TÓM TẮT NỘI DUNG LUẬN VĂN}

Luận văn này nghiên cứu sự tồn tại và đặc trưng của các giới hạn lỗi toàn cục cho các hàm lồi trong không gian Euclid, cung cấp các điều kiện cần và đủ mà không cần áp đặt các yêu cầu bổ sung lên các hàm hoặc tập nghiệm của chúng. Bằng cách tập trung vào không gian Euclid, chúng tôi đưa ra các điều kiện dễ kiểm tra, nâng cao tính ứng dụng thực tế.

Hơn nữa, chúng tôi khám phá các tính chất của đạo hàm dưới của hàm cực đại được hình thành từ một tập hợp tùy ý của các hàm lồi liên tục. Cụ thể, chúng tôi chứng minh rằng, dưới một giả định thích hợp, đạo hàm dưới của hàm cực đại này tương ứng với bao lồi của các đạo hàm dưới của các hàm thành phần tại các chỉ số đang hoạt động. Kết quả này mở rộng hiểu biết của chúng ta về mối quan hệ giữa các đạo hàm dưới của các hàm thành phần và hàm cực đại của chúng, cung cấp một góc nhìn mới về phân tích lồi.

Các ứng dụng của những phát hiện lý thuyết này đặc biệt có liên quan đến việc nghiên cứu các giới hạn lỗi toàn cục trong các hệ thống được đặc trưng bởi các bất đẳng thức tuyến tính và lồi vô hạn. Kết quả của chúng tôi không chỉ tổng quát hóa các lý thuyết hiện có mà còn cung cấp các tiêu chí thực tế dễ áp dụng hơn.

Xuyên suốt luận văn, nhiều ví dụ được đưa ra để minh họa những ưu điểm của các điều kiện đề xuất so với các điều kiện hiện có, chứng minh hiệu quả và tính hữu dụng cải thiện của phương pháp tiếp cận của chúng tôi trong nhiều ngữ cảnh khác nhau. Các ví dụ này nhấn mạnh tầm quan trọng thực tiễn của những đóng góp lý thuyết của chúng tôi, làm nổi bật tiềm năng ảnh hưởng đến việc giải quyết các vấn đề tối ưu hóa phức tạp liên quan đến các hàm lồi.

\section*{2. NHỮNG KẾT QUẢ MỚI CỦA LUẬN VĂN}

Luận văn này phát triển lý thuyết về đạo hàm dưới và đạo hàm dưới riêng biệt mà không cần các giả định về tính đóng hoặc tính nửa liên tục dưới của các tập hợp và hàm số. Điều này đánh dấu một bước tiến quan trọng so với các nghiên cứu trước đây, nơi mà các giả định này thường được xem là cần thiết. Bằng cách loại bỏ các yêu cầu khắt khe này, luận văn mở ra những cách tiếp cận linh hoạt hơn cho việc phân tích và ứng dụng các hàm lồi. Kết quả là, lý thuyết phát triển trong luận văn không chỉ dễ kiểm tra và áp dụng hơn mà còn mở rộng phạm vi của các bài toán tối ưu hóa có thể giải quyết, đặc biệt trong bối cảnh các hệ thống bất đẳng thức tuyến tính và lồi vô hạn.

\section*{3. CÁC ỨNG DỤNG/ KHẢ NĂNG ỨNG DỤNG TRONG THỰC TIỄN HAY NHỮNG VẤN ĐỀ CÒN BỎ NGỎ CẦN TIẾP TỤC NGHIÊN CỨU}

Các ứng dụng của luận văn này đặc biệt có ảnh hưởng trong việc nghiên cứu các bài toán tối ưu hóa trong không gian Banach, tập trung vào tính chất Pshenichnyi-Levin-Valadier (PLV). Bằng cách thiết lập các điều kiện cần và đủ cho tính chất PLV mà không cần các giả định về tính đóng và tính nửa liên tục dưới, luận văn cung cấp một khung lý thuyết vững chắc để phân tích và giải quyết các bài toán tối ưu hóa phức tạp trong các không gian này. Điều này góp phần vào việc hiểu sâu hơn về các tính chất cấu trúc của các hàm lồi và bất đẳng thức trong không gian Banach, nâng cao khả năng giải quyết nhiều bài toán thực tế hơn. Các phát hiện này có ý nghĩa quan trọng đối với các lĩnh vực đòi hỏi kỹ thuật tối ưu hóa phức tạp, chẳng hạn như kinh tế, kỹ thuật và toán học ứng dụng, nơi tính chất PLV đóng vai trò quan trọng trong việc mô hình hóa và giải quyết các vấn đề thực tiễn.

\vspace{4\baselineskip}
\begin{table}[H]
\begin{adjustbox}{max width =\textwidth}
\begin{tabular}{p{8.44cm}p{8.4cm}}
\multicolumn{1}{p{8.44cm}}{
\centering \textbf{TẬP THỂ CÁN BỘ HƯỚNG DẪN} \newline
\centering
(Ký tên, họ tên) \newline
} &
\multicolumn{1}{p{8.4cm}}{
\centering \textbf{HỌC VIÊN CAO HỌC} \newline
\centering
(Ký tên, họ tên) \newline
} \\
\end{tabular}
\end{adjustbox}
\end{table}
\vspace{2\baselineskip}
\begin{center}
    \textbf{XÁC NHẬN CỦA CƠ SỞ ĐÀO TẠO}
\end{center}
\begin{center}
    \textbf{HIỆU TRƯỞNG}
\end{center}