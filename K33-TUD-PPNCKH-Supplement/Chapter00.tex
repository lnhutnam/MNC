\chapter*{LỜI NÓI ĐẦU}
\addcontentsline{toc}{chapter}{LỜI NÓI ĐẦU}

Trong luận văn này, chúng tôi nghiên cứu các điều kiện cần và đủ cho sự tồn tại của các chặn sai số toàn cục cho một hàm lồi mà \emph{không cần} các điều kiện bổ trợ trên không gian hàm hay không gian nghiệm. Nói một cách cụ thể, chúng tôi đưa ra các đặc trưng cho các chặn sai số trong không gian Euclidean thông qua một số kiểm tra đơn giản. Kế tiếp, chúng tôi chứng minh dưới một giả định phù hợp rằng dưới vi phân của hàm chặn trên nhỏ nhất của một họ bất kỳ các hàm liên tục lồi trùng với bao lồi của dưới vi phân của các hàm tương ứng với chi số dương tại các điểm cho trước. Cuối cùng, chúng tôi nghiên cứu sự tồn tại của các chặn sai số toàn cục cho hệ vô hạn các bất đẳng thức tuyến tính và lồi. 

Nội dung luận văn bao gồm bốn chương:

\begin{description}
    \item [Chương 1: Giới thiệu] Trình bày nội dung chương 1.
    \item [Chương 2: Kiến thức chuẩn bị] Trình bày các kiến thức chuẩn bị để từ đó làm
tiền đề cho toàn bộ luận văn.
    \item [Chương 3: Chặn sai số toàn cục cho một hệ bất phương trình lồi] Trình bày một số điều kiện cần và đủ cho sự tồn tại của chặn sai số toàn cục của một hệ bất phương trình lồi. 
    \item [Chương 4: Điều kiện cần và đủ cho tính chất PLV, và ứng dụng] Trình bày quy tắc để tính toán tập hợp dưới vi phân của chặn trên nhỏ nhất từng điểm của hàm lồi liên tục được định nghĩa trên không gian Banach. Từ đó, áp dụng để khám phá sự tồn tại của chặn sai số toàn cục cho hệ vô hạn bất phương trình lồi và tuyến tính.
\end{description}
