% Input configuration
\usepackage[utf8]{inputenc}

% -- support vietnamese --
% \usepackage[vietnamese]{babel}
% -- support english --
\usepackage[english]{babel}
% ----------------------------------------------

% For math
\usepackage{amsmath,amssymb}
\usepackage{mathtools}
\usepackage{amsthm}
% \newcommand{\N}{\mathbb{N}}
% \newcommand{\Z}{\mathbb{Z}}
% \newcommand{\R}{\mathbb{R}}
% \newcommand{\F}{\mathbb{F}}
% ----------------------------------------------

% For images, colors
\usepackage{graphicx}
\usepackage{color,xcolor}
% ----------------------------------------------

% For captioning
\usepackage{caption}
\setbeamertemplate{caption}[numbered]
% ----------------------------------------------

% For tables
\usepackage{booktabs}
% ----------------------------------------------

% For hyper references between sections
\usepackage{hyperref}
% ----------------------------------------------

% For texts, algorithm and codes
\usepackage{ulem}
\usepackage{algorithm}
\usepackage{listings}
% ----------------------------------------------

% For tikz
\usepackage{framed}
\usepackage{tikz}
\usepackage{pgf}
\usetikzlibrary{calc,trees,positioning,arrows,chains,shapes.geometric,%
decorations.pathreplacing,decorations.pathmorphing,shapes,%
matrix,shapes.symbols}
\pgfmathsetseed{1} % To have predictable results
% Define a background layer, in which the parchment shape is drawn
\definecolor{doge}{HTML}{0065BD}
\pgfdeclarelayer{background}
\pgfsetlayers{background, main}

% Macro to draw the shape behind the text, when it fits completly in the page
% define styles for the normal border and the torn border
\tikzset{
  normal border/.style={doge!40!black!10, decorate, 
     decoration={random steps, segment length=2.5cm, amplitude=.7mm}},
  torn border/.style={doge!40!black!5, decorate, 
     decoration={random steps, segment length=.5cm, amplitude=1.7mm}}}

% Macro to draw the shape behind the text, when it fits completly in the
% page
\def\parchmentframe#1{
\tikz{
  \node[inner sep=2em] (A) {#1};  % Draw the text of the node
  \begin{pgfonlayer}{background}  % Draw the shape behind
  \fill[normal border] 
        (A.south east) -- (A.south west) -- 
        (A.north west) -- (A.north east) -- cycle;
  \end{pgfonlayer}}}

% Macro to draw the shape, when the text will continue in next page
\def\parchmentframetop#1{
\tikz{
  \node[inner sep=2em] (A) {#1};    % Draw the text of the node
  \begin{pgfonlayer}{background}    
  \fill[normal border]              % Draw the ``complete shape'' behind
        (A.south east) -- (A.south west) -- 
        (A.north west) -- (A.north east) -- cycle;
  \fill[torn border]                % Add the torn lower border
        ($(A.south east)-(0,.2)$) -- ($(A.south west)-(0,.2)$) -- 
        ($(A.south west)+(0,.2)$) -- ($(A.south east)+(0,.2)$) -- cycle;
  \end{pgfonlayer}}}

% Macro to draw the shape, when the text continues from previous page
\def\parchmentframebottom#1{
\tikz{
  \node[inner sep=2em] (A) {#1};   % Draw the text of the node
  \begin{pgfonlayer}{background}   
  \fill[normal border]             % Draw the ``complete shape'' behind
        (A.south east) -- (A.south west) -- 
        (A.north west) -- (A.north east) -- cycle;
  \fill[torn border]               % Add the torn upper border
        ($(A.north east)-(0,.2)$) -- ($(A.north west)-(0,.2)$) -- 
        ($(A.north west)+(0,.2)$) -- ($(A.north east)+(0,.2)$) -- cycle;
  \end{pgfonlayer}}}

% Macro to draw the shape, when both the text continues from previous page
% and it will continue in next page
\def\parchmentframemiddle#1{
\tikz{
  \node[inner sep=2em] (A) {#1};   % Draw the text of the node
  \begin{pgfonlayer}{background}   
  \fill[normal border]             % Draw the ``complete shape'' behind
        (A.south east) -- (A.south west) -- 
        (A.north west) -- (A.north east) -- cycle;
  \fill[torn border]               % Add the torn lower border
        ($(A.south east)-(0,.2)$) -- ($(A.south west)-(0,.2)$) -- 
        ($(A.south west)+(0,.2)$) -- ($(A.south east)+(0,.2)$) -- cycle;
  \fill[torn border]               % Add the torn upper border
        ($(A.north east)-(0,.2)$) -- ($(A.north west)-(0,.2)$) -- 
        ($(A.north west)+(0,.2)$) -- ($(A.north east)+(0,.2)$) -- cycle;
  \end{pgfonlayer}}}

\newenvironment{parchment}[1][Example]{%
  \def\FrameCommand{\parchmentframe}%
  \def\FirstFrameCommand{\parchmentframetop}%
  \def\LastFrameCommand{\parchmentframebottom}%
  \def\MidFrameCommand{\parchmentframemiddle}%
  \vskip\baselineskip
  \MakeFramed {\FrameRestore}
  \noindent\tikz\node[inner sep=1ex, draw=white,fill=doge!30, 
          anchor=west, overlay] at (0em, 2em) {\sffamily#1};\par}%
{\endMakeFramed}
% ----------------------------------------------

% Beamer theme configuration
\mode<presentation>{
    \usetheme{default}
    % Boadilla CambridgeUS
    % default Antibes Berlin Copenhagen
    % Madrid Montpelier Ilmenau Malmoe
    % Berkeley Singapore Warsaw
    % \usecolortheme{doge}
    % beetle, beaver, orchid, whale, dolphin
    \useoutertheme{infolines}
    % infolines miniframes shadow sidebar smoothbars smoothtree split tree
    \useinnertheme{circles}
    % circles, rectanges, rounded, inmargin
}

\setbeamertemplate{blocks}[rounded][shadow]
\setbeamercolor{block title}{bg=doge!40,fg=white}
\newcommand{\reditem}[1]{\setbeamercolor{item}{fg=red}\item #1}
\newcommand*{\Scale}[2][4]{\scalebox{#1}{\ensuremath{#2}}}
\renewcommand\textbullet{\ensuremath{\bullet}}
% ----------------------------------------------

% For flow chart and codes
% Flow chart settings
\tikzset{
    >=stealth',
    punktchain/.style={
        rectangle, 
        rounded corners, 
        % fill=black!10,
        draw=white, very thick,
        text width=6em,
        minimum height=2em, 
        text centered, 
        on chain
    },
    largepunktchain/.style={
        rectangle,
        rounded corners,
        draw=white, very thick,
        text width=10em,
        minimum height=2em,
        on chain
    },
    line/.style={draw, thick, <-},
    element/.style={
        tape,
        top color=white,
        bottom color=blue!50!black!60!,
        minimum width=6em,
        draw=blue!40!black!90, very thick,
        text width=6em, 
        minimum height=2em, 
        text centered, 
        on chain
    },
    every join/.style={->, thick,shorten >=1pt},
    decoration={brace},
    tuborg/.style={decorate},
    tubnode/.style={midway, right=2pt},
    font={\fontsize{10pt}{12}\selectfont},
}

% code settings
\lstset{
    language=C++,
    basicstyle=\ttfamily\footnotesize,
    keywordstyle=\color{red},
    breaklines=true,
    xleftmargin=2em,
    numbers=left,
    numberstyle=\color[RGB]{222,155,81},
    frame=leftline,
    tabsize=4,
    breakatwhitespace=false,
    showspaces=false,               
    showstringspaces=false,
    showtabs=false,
    morekeywords={Str, Num, List},
}
% ---------------------------------------------------------------------
% title page
\makeatletter
\newcommand\titlegraphicii[1]{\def\inserttitlegraphicii{#1}}
\titlegraphicii{}
\setbeamertemplate{title page}
{
    \vskip-0.5cm%
  \vbox{}
   {\usebeamercolor[fg]{titlegraphic}\inserttitlegraphic\hfill\inserttitlegraphicii\par}
  \begin{centering}
    \begin{beamercolorbox}[sep=8pt,center]{title}
      \usebeamerfont{title}\inserttitle\par%
      \vskip0.5em
      \ifx\insertsubtitle\@empty%
      \else%
        \vskip0.25em%
        {\usebeamerfont{subtitle}\usebeamercolor[fg]{subtitle}\insertsubtitle\par}%
      \fi%     
    \end{beamercolorbox}%
    \begin{beamercolorbox}[sep=8pt,center]{author}
      \usebeamerfont{author}\insertauthor
    \end{beamercolorbox}\vskip0.5em
    \begin{beamercolorbox}[sep=8pt,center]{institute}
      \usebeamerfont{institute}\insertinstitute
    \end{beamercolorbox}
    \vskip1em\par
    \begin{beamercolorbox}[sep=8pt,center]{date}
      \usebeamerfont{date}\insertdate
    \end{beamercolorbox}%
  \end{centering}
  %\vfill
}

\makeatother
\author{Presented by Author}
\title{Presentation Title}
\subtitle{Presentation Subtile}
\institute{Department \\ University}
\date{\today}

% frame title
% https://tex.stackexchange.com/questions/231554/set-beamercolorbox-height-automatically-to-sister-beamercolorbox-on-frametitle
\makeatletter
\setbeamertemplate{frametitle}{%
    \setbeamercolor{frametitle}{bg=doge, fg=white}
    \nointerlineskip%
    \usebeamerfont{headline}%
    \nointerlineskip%
    \hbox{\hspace{-0.09\paperwidth}%
    \begin{beamercolorbox}[wd=0.1\paperwidth,vmode]{secsubsec}%
        \newdimen\titleheight%
        \setbox0=\hbox{\usebeamerfont*{frametitle}\insertframetitle}
        \titleheight=\ht0 \advance\titleheight by \dp0%
        \vskip-.5pt%
        \vskip\titleheight%
        \ifx\insertframesubtitle\@empty%
          \strut\par%
        \else%
          \setbox0=\hbox{\usebeamerfont*{framesubtitle}\insertframesubtitle}%
          \titleheight=\ht0 \advance\titleheight by \dp0%
          \par{%
              \vskip\titleheight%
              \strut\par%
              \vskip-.65ex%
          }%
        \fi%
        \usebeamerfont{headline}%
        \vskip.5ex%
    \end{beamercolorbox}%
    \begin{beamercolorbox}[wd=0.99\paperwidth,leftskip=.3cm,rightskip=.3cm plus1fil,vmode]{frametitle}%
        \vskip.5ex%
        \usebeamerfont*{frametitle}\insertframetitle%
        \ifx\insertframesubtitle\@empty%
          \strut\par%
        \else%
          \par{\usebeamerfont*{framesubtitle}{\usebeamercolor[fg]{framesubtitle}\insertframesubtitle}\strut\par}%
        \fi%%
        \usebeamerfont{headline}%
        \vskip.5ex%
    \end{beamercolorbox}%
    }
    \nointerlineskip
}

% footer
\makeatother
\setbeamertemplate{footline}
{
  \leavevmode%
  \hbox{%
  \begin{beamercolorbox}[wd=.3\paperwidth,ht=2.25ex,dp=1ex,center]{author in head/foot}%
    \usebeamerfont{author in head/foot}\insertshortauthor
  \end{beamercolorbox}%
  \begin{beamercolorbox}[wd=.4\paperwidth,ht=2.25ex,dp=1ex,center]{title in head/foot}%
    \usebeamerfont{title in head/foot}\insertshorttitle
  \end{beamercolorbox}}%
  \begin{beamercolorbox}[wd=.3\paperwidth,ht=2.25ex,dp=1ex,center]{pagenum in head/foot}%
    \insertframenumber{} / \inserttotalframenumber\hspace*{1ex}
  \end{beamercolorbox}%
  \vskip0pt%
}
\makeatletter
\setbeamertemplate{navigation symbols}{}

\setbeamertemplate{section page}
{
    \begin{centering}
    \begin{beamercolorbox}[sep=12pt,center]{part title}
    \usebeamerfont{section title}\insertsection\par
    \end{beamercolorbox}
    \end{centering}
}

\AtBeginSection[]
{
    \setbeamertemplate{navigation symbols}{}
    \frame[plain,c,noframenumbering]{
        \sectionpage
        \tableofcontents[currentsection,subsectionstyle=hide]}
    \setbeamertemplate{navigation symbols}{\normalsize}
}

\newtheorem{defi}{Definition}
\newtheorem{theorem}{Theorem}[section]
\newtheorem{lemma}{Bổ đề}
\newtheorem{prop}{Proposition}
\newtheorem{eg}{Example}


\usetikzlibrary{arrows}
\usetikzlibrary{arrows.meta}
\usetikzlibrary{decorations.markings}
\usetikzlibrary{decorations.pathmorphing}
\usetikzlibrary{positioning}
\usetikzlibrary{fadings}
\usetikzlibrary{intersections}
\usetikzlibrary{cd}

\let\U\relax
\let\C\relax
\let\G\relax

% Matrix groups
\newcommand{\GL}{\mathrm{GL}}
\newcommand{\Or}{\mathrm{O}}
\newcommand{\PGL}{\mathrm{PGL}}
\newcommand{\PSL}{\mathrm{PSL}}
\newcommand{\PSO}{\mathrm{PSO}}
\newcommand{\PSU}{\mathrm{PSU}}
\newcommand{\SL}{\mathrm{SL}}
\newcommand{\SO}{\mathrm{SO}}
\newcommand{\Spin}{\mathrm{Spin}}
\newcommand{\Sp}{\mathrm{Sp}}
\newcommand{\SU}{\mathrm{SU}}
\newcommand{\U}{\mathrm{U}}
\newcommand{\Mat}{\mathrm{Mat}}

% Matrix algebras
\newcommand{\gl}{\mathfrak{gl}}
\newcommand{\ort}{\mathfrak{o}}
% \newcommand{\so}{\mathfrak{so}}
\newcommand{\su}{\mathfrak{su}}
\newcommand{\uu}{\mathfrak{u}}
\renewcommand{\sl}{\mathfrak{sl}}

% Special sets
\newcommand{\C}{\mathbb{C}}
\newcommand{\CP}{\mathbb{CP}}
\newcommand{\GG}{\mathbb{G}}
\newcommand{\N}{\mathbb{N}}
\newcommand{\Q}{\mathbb{Q}}
\newcommand{\R}{\mathbb{R}}
\newcommand{\RP}{\mathbb{RP}}
\newcommand{\T}{\mathbb{T}}
\newcommand{\Z}{\mathbb{Z}}
\renewcommand{\H}{\mathbb{H}}

% Brackets
\newcommand{\abs}[1]{\left\lvert #1\right\rvert}
\newcommand{\bket}[1]{\left\lvert #1\right\rangle}
\newcommand{\brak}[1]{\left\langle #1 \right\rvert}
\newcommand{\braket}[2]{\left\langle #1\middle\vert #2 \right\rangle}
\newcommand{\bra}{\langle}
\newcommand{\ket}{\rangle}
\newcommand{\norm}[1]{\left\lVert #1\right\rVert}
\newcommand{\normalorder}[1]{\mathop{:}\nolimits\!#1\!\mathop{:}\nolimits}
\newcommand{\tv}[1]{|#1|}
\renewcommand{\vec}[1]{\boldsymbol{\mathbf{#1}}}

% not-math
\newcommand{\bolds}[1]{{\bfseries #1}}
\newcommand{\cat}[1]{\mathsf{#1}}
\newcommand{\ph}{\,\cdot\,}
\newcommand{\term}[1]{\emph{#1}\index{#1}}
\newcommand{\phantomeq}{\hphantom{{}={}}}
% Probability
\DeclareMathOperator{\Bernoulli}{Bernoulli}
\DeclareMathOperator{\betaD}{beta}
\DeclareMathOperator{\bias}{bias}
\DeclareMathOperator{\binomial}{binomial}
\DeclareMathOperator{\corr}{corr}
\DeclareMathOperator{\cov}{cov}
\DeclareMathOperator{\gammaD}{gamma}
\DeclareMathOperator{\mse}{mse}
\DeclareMathOperator{\multinomial}{multinomial}
\DeclareMathOperator{\Poisson}{Poisson}
\DeclareMathOperator{\var}{var}
\newcommand{\E}{\mathbb{E}}
\newcommand{\Prob}{\mathbb{P}}

% Algebra
\DeclareMathOperator{\adj}{adj}
\DeclareMathOperator{\Ann}{Ann}
\DeclareMathOperator{\Aut}{Aut}
\DeclareMathOperator{\Char}{char}
\DeclareMathOperator{\disc}{disc}
\DeclareMathOperator{\dom}{dom}
\DeclareMathOperator{\fix}{fix}
\DeclareMathOperator{\Hom}{Hom}
\DeclareMathOperator{\id}{id}
\DeclareMathOperator{\image}{image}
\DeclareMathOperator{\im}{im}
\DeclareMathOperator{\re}{re}
\DeclareMathOperator{\tr}{tr}
\DeclareMathOperator{\Tr}{Tr}
\newcommand{\Bilin}{\mathrm{Bilin}}
\newcommand{\Frob}{\mathrm{Frob}}

% Others
\newcommand\ad{\mathrm{ad}}
\newcommand\Art{\mathrm{Art}}
\newcommand{\B}{\mathcal{B}}
\newcommand{\cU}{\mathcal{U}}
\newcommand{\Der}{\mathrm{Der}}
\newcommand{\D}{\mathrm{D}}
\newcommand{\dR}{\mathrm{dR}}
\newcommand{\exterior}{\mathchoice{{\textstyle\bigwedge}}{{\bigwedge}}{{\textstyle\wedge}}{{\scriptstyle\wedge}}}
\newcommand{\F}{\mathbb{F}}
\newcommand{\G}{\mathcal{G}}
\newcommand{\Gr}{\mathrm{Gr}}
\newcommand{\haut}{\mathrm{ht}}
\newcommand{\Hol}{\mathrm{Hol}}
\newcommand{\hol}{\mathfrak{hol}}
\newcommand{\Id}{\mathrm{Id}}
\newcommand{\lie}[1]{\mathfrak{#1}}
\newcommand{\op}{\mathrm{op}}
\newcommand{\Oc}{\mathcal{O}}
\newcommand{\pr}{\mathrm{pr}}
\newcommand{\Ps}{\mathcal{P}}
\newcommand{\pt}{\mathrm{pt}}
\newcommand{\qeq}{\mathrel{``{=}"}}
\newcommand{\Rs}{\mathcal{R}}
\newcommand{\Vect}{\mathrm{Vect}}
\newcommand{\wsto}{\stackrel{\mathrm{w}^*}{\to}}
\newcommand{\wt}{\mathrm{wt}}
\newcommand{\wto}{\stackrel{\mathrm{w}}{\to}}
\renewcommand{\d}{\mathrm{d}}
\renewcommand{\P}{\mathbb{P}}
%\renewcommand{\F}{\mathcal{F}}


\let\Im\relax
\let\Re\relax

\DeclareMathOperator{\area}{area}
\DeclareMathOperator{\card}{card}
\DeclareMathOperator{\ccl}{ccl}
\DeclareMathOperator{\ch}{ch}
\DeclareMathOperator{\cl}{cl}
\DeclareMathOperator{\cls}{\overline{\mathrm{span}}}
\DeclareMathOperator{\coker}{coker}
\DeclareMathOperator{\conv}{conv}
\DeclareMathOperator{\cosec}{cosec}
\DeclareMathOperator{\cosech}{cosech}
\DeclareMathOperator{\covol}{covol}
\DeclareMathOperator{\diag}{diag}
\DeclareMathOperator{\diam}{diam}
\DeclareMathOperator{\Diff}{Diff}
\DeclareMathOperator{\End}{End}
\DeclareMathOperator{\energy}{energy}
\DeclareMathOperator{\erfc}{erfc}
\DeclareMathOperator{\erf}{erf}
\DeclareMathOperator*{\esssup}{ess\,sup}
\DeclareMathOperator{\ev}{ev}
\DeclareMathOperator{\Ext}{Ext}
\DeclareMathOperator{\fst}{fst}
\DeclareMathOperator{\Fit}{Fit}
\DeclareMathOperator{\Frac}{Frac}
\DeclareMathOperator{\Gal}{Gal}
\DeclareMathOperator{\gr}{gr}
\DeclareMathOperator{\hcf}{hcf}
\DeclareMathOperator{\Im}{Im}
\DeclareMathOperator{\Ind}{Ind}
\DeclareMathOperator{\Int}{Int}
\DeclareMathOperator{\Isom}{Isom}
\DeclareMathOperator{\lcm}{lcm}
\DeclareMathOperator{\length}{length}
\DeclareMathOperator{\Lie}{Lie}
\DeclareMathOperator{\like}{like}
\DeclareMathOperator{\Lk}{Lk}
\DeclareMathOperator{\Maps}{Maps}
\DeclareMathOperator{\orb}{orb}
\DeclareMathOperator{\ord}{ord}
\DeclareMathOperator{\otp}{otp}
\DeclareMathOperator{\poly}{poly}
\DeclareMathOperator{\rank}{rank}
\DeclareMathOperator{\rel}{rel}
\DeclareMathOperator{\Rad}{Rad}
\DeclareMathOperator{\Re}{Re}
\DeclareMathOperator*{\res}{res}
\DeclareMathOperator{\Res}{Res}
\DeclareMathOperator{\Ric}{Ric}
\DeclareMathOperator{\rk}{rk}
\DeclareMathOperator{\Rees}{Rees}
\DeclareMathOperator{\Root}{Root}
\DeclareMathOperator{\sech}{sech}
\DeclareMathOperator{\sgn}{sgn}
\DeclareMathOperator{\snd}{snd}
\DeclareMathOperator{\Spec}{Spec}
\DeclareMathOperator{\spn}{span}
\DeclareMathOperator{\stab}{stab}
\DeclareMathOperator{\St}{St}
\DeclareMathOperator{\supp}{supp}
\DeclareMathOperator{\Syl}{Syl}
\DeclareMathOperator{\Sym}{Sym}
\DeclareMathOperator{\vol}{vol}

\pgfarrowsdeclarecombine{twolatex'}{twolatex'}{latex'}{latex'}{latex'}{latex'}
\tikzset{->/.style = {decoration={markings,
                                  mark=at position 1 with {\arrow[scale=2]{latex'}}},
                      postaction={decorate}}}
\tikzset{<-/.style = {decoration={markings,
                                  mark=at position 0 with {\arrowreversed[scale=2]{latex'}}},
                      postaction={decorate}}}
\tikzset{<->/.style = {decoration={markings,
                                   mark=at position 0 with {\arrowreversed[scale=2]{latex'}},
                                   mark=at position 1 with {\arrow[scale=2]{latex'}}},
                       postaction={decorate}}}
\tikzset{->-/.style = {decoration={markings,
                                   mark=at position #1 with {\arrow[scale=2]{latex'}}},
                       postaction={decorate}}}
\tikzset{-<-/.style = {decoration={markings,
                                   mark=at position #1 with {\arrowreversed[scale=2]{latex'}}},
                       postaction={decorate}}}
\tikzset{->>/.style = {decoration={markings,
                                  mark=at position 1 with {\arrow[scale=2]{latex'}}},
                      postaction={decorate}}}
\tikzset{<<-/.style = {decoration={markings,
                                  mark=at position 0 with {\arrowreversed[scale=2]{twolatex'}}},
                      postaction={decorate}}}
\tikzset{<<->>/.style = {decoration={markings,
                                   mark=at position 0 with {\arrowreversed[scale=2]{twolatex'}},
                                   mark=at position 1 with {\arrow[scale=2]{twolatex'}}},
                       postaction={decorate}}}
\tikzset{->>-/.style = {decoration={markings,
                                   mark=at position #1 with {\arrow[scale=2]{twolatex'}}},
                       postaction={decorate}}}
\tikzset{-<<-/.style = {decoration={markings,
                                   mark=at position #1 with {\arrowreversed[scale=2]{twolatex'}}},
                       postaction={decorate}}}

\tikzset{circ/.style = {fill, circle, inner sep = 0, minimum size = 3}}
\tikzset{scirc/.style = {fill, circle, inner sep = 0, minimum size = 1.5}}
\tikzset{mstate/.style={circle, draw, blue, text=black, minimum width=0.7cm}}

\tikzset{eqpic/.style={baseline={([yshift=-.5ex]current bounding box.center)}}}
\tikzset{commutative diagrams/.cd,cdmap/.style={/tikz/column 1/.append style={anchor=base east},/tikz/column 2/.append style={anchor=base west},row sep=tiny}}

\definecolor{mblue}{rgb}{0.2, 0.3, 0.8}
\definecolor{morange}{rgb}{1, 0.5, 0}
\definecolor{mgreen}{rgb}{0.1, 0.4, 0.2}
\definecolor{mred}{rgb}{0.5, 0, 0}

\def\drawcirculararc(#1,#2)(#3,#4)(#5,#6){%
    \pgfmathsetmacro\cA{(#1*#1+#2*#2-#3*#3-#4*#4)/2}%
    \pgfmathsetmacro\cB{(#1*#1+#2*#2-#5*#5-#6*#6)/2}%
    \pgfmathsetmacro\cy{(\cB*(#1-#3)-\cA*(#1-#5))/%
                        ((#2-#6)*(#1-#3)-(#2-#4)*(#1-#5))}%
    \pgfmathsetmacro\cx{(\cA-\cy*(#2-#4))/(#1-#3)}%
    \pgfmathsetmacro\cr{sqrt((#1-\cx)*(#1-\cx)+(#2-\cy)*(#2-\cy))}%
    \pgfmathsetmacro\cA{atan2(#2-\cy,#1-\cx)}%
    \pgfmathsetmacro\cB{atan2(#6-\cy,#5-\cx)}%
    \pgfmathparse{\cB<\cA}%
    \ifnum\pgfmathresult=1
        \pgfmathsetmacro\cB{\cB+360}%
    \fi
    \draw (#1,#2) arc (\cA:\cB:\cr);%
}
\newcommand\getCoord[3]{\newdimen{#1}\newdimen{#2}\pgfextractx{#1}{\pgfpointanchor{#3}{center}}\pgfextracty{#2}{\pgfpointanchor{#3}{center}}}

\newcommand\qedshift{\vspace{-17pt}}
\newcommand\fakeqed{\pushQED{\qed}\qedhere}

\def\Xint#1{\mathchoice
   {\XXint\displaystyle\textstyle{#1}}%
   {\XXint\textstyle\scriptstyle{#1}}%
   {\XXint\scriptstyle\scriptscriptstyle{#1}}%
   {\XXint\scriptscriptstyle\scriptscriptstyle{#1}}%
   \!\int}
\def\XXint#1#2#3{{\setbox0=\hbox{$#1{#2#3}{\int}$}
     \vcenter{\hbox{$#2#3$}}\kern-.5\wd0}}
\def\ddashint{\Xint=}
\def\dashint{\Xint-}

\newcommand\separator{{\centering\rule{2cm}{0.2pt}\vspace{2pt}\par}}

\newenvironment{own}{\color{gray!70!black}}{}

\newcommand\makecenter[1]{\raisebox{-0.5\height}{#1}}

\mathchardef\mdash="2D

\newenvironment{significant}{\begin{center}\begin{minipage}{0.9\textwidth}\centering\em}{\end{minipage}\end{center}}
\DeclareRobustCommand{\rvdots}{%
  \vbox{
    \baselineskip4\p@\lineskiplimit\z@
    \kern-\p@
    \hbox{.}\hbox{.}\hbox{.}
  }}
\DeclareRobustCommand\tph[3]{{\texorpdfstring{#1}{#2}}}
\makeatother