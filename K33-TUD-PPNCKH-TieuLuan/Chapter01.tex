\chapter*{LỜI NÓI ĐẦU}
\addcontentsline{toc}{chapter}{LỜI NÓI ĐẦU}

Đây là tiểu luận cuối kỳ môn học Phương pháp nghiên cứu khoa học, cao học ngành Toán, khóa 33. Trong tiểu luận này, chúng tôi trình bày thành từng chương, mỗi chương bình luận và đưa ra câu trả lời cho từng câu hỏi mà đề tài yêu cầu. Hai chương đầu tiên của tiểu luận thuộc phần thực hành, và các chương còn lại thuộc phần lý thuyết. Cấu trúc của tiển luận như sau: 
\begin{description}
\item [Chương 1: Tóm tắt và nhận xét khoa học] Trong chương này, chúng tôi thực hành làm một bản tóm tắt khoa học và một bản nhận xét khoa học đối với một bài báo khoa học đã được chấp nhận đăng tải trên một tạp chí ngành Toán. 
\item [Chương 2: Tập hợp, Hàm, và Quan hệ] Trong chương này, chúng tôi trình bày một chủ đề đầu tiên và cũng là nền tảng cho Toán cao cấp - Tập hợp, Hàm, và Quan hệ.  
\item [Chương 3: Phân tích về tri thức kinh nghiệm \& tri thức khoa học] Đối với chương này, chúng tôi trình bày và phân tích về tri thức kinh nghiệm và tri thức khoa học để làm rõ sự khác biệt giữa hai khái niệm này. 
\item [Chương 4: Mục đích và mục tiêu nghiên cứu] Trong chương này, chúng tôi trình bày và phân tích về mục đích và mục tiêu trong nghiên cứu khoa học. Từ đó, chúng tôi chỉ ra và làm rõ sự khác biệt giữa hai khái niệm này.
\item [Chương 5: Phẩm chất trong nghiên cứu khoa học] Trong chương này, chúng tôi trình bày và phân tích những phẩm chất trong một nghiên cứu khoa học. Chúng tôi dựa trên những nhận định của GS. Ngô Bảo Châu và tập trung vào ba yếu tố chính bao gồm: đúng, trung thực, và mới và hay, trong đó tập trung nhấn mạnh vào yếu tố đúng và trung thực.
\item [Chương 6: Lựa chọn đề tài nghiên cứu khoa học] Trong chương này, chúng tôi trình và phân tích về giai đoạn lựa chọn đề tài nghiên cứu khoa học.
\item [Chương 7: Vai trò của tạp chí Mathematical reviews] Trong chương này, chúng tôi tìm hiểu và thảo luận về vai trò của tạp chí Mathematical reviews đối với những người làm Toán.
\item [Chương 8: Bình luận về Impact Factor] Chỉ số Impact Factor (IF) là một trong các chỉ số quan trọng trong việc lựa chọn tạp chí. Vì thế, trong chương này chúng tôi trình bày về nó, và phân tích ý nghĩa của nó đối với một tạp chí. Đồng thời, chúng tôi cũng đặt ra và làm rõ một câu hỏi "liệu Impact factor thấp thì có đáng lo ngại?".
\item [Chương 9: Thông tin học viên] Có lẽ gọi là chương thì cũng không đúng lắm, nhưng để phù hợp và thống nhất với cấu trúc chung mà ban đầu chúng tôi đã đặt ra thì chúng tôi tạm gọi nó là Chương. Trong chương này trình bày về một số thông tin cơ bản của học viên và ngành học của học viên.
\item [Chương 10: Thiết kế luận văn thạc sĩ Toán học] Trong chương này, chúng tôi thực hiện trình bày cấu trúc hoàn chỉnh cho một luận văn thạc sĩ Toán học với một đề tài giả định.
\end{description}
