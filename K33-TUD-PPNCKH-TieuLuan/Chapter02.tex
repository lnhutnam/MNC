\chapter{TÓM TẮT VÀ NHẬN XÉT KHOA HỌC}

Trong chương này, chúng tôi thực hành làm một bản tóm tắt và nhận xét khoa học đối với bài báo \emph{On Global Error Bounds for Convex Inequalities Systems} của tác giả/ Thầy Võ Sĩ Trọng Long đăng trên tạp chí \emph{Journal of Optimization Theory and Applications}. Bản đính kèm copy được ghép trong báo cáo này.

\section{Bản tóm tắt khoa học}

Trong bài báo này, trước tiên tác giả trình bày các điều kiện cần và đủ cho sự tồn tại của chặn sai số toàn cục đối với hàm lồi mà không có điều kiện bổ sung đối với hàm hoặc tập nghiệm. Đặc biệt, chúng ta thu được các đặc tính của các giới hạn sai số toàn cục như vậy trong không gian Euclide, thường dễ kiểm tra. Thứ hai, chúng tôi cung cấp
rằng theo một giả định phù hợp, dưới vi phân của hàm tối cao của một họ tùy ý các hàm liên tục lồi trùng với bao lồi của dưới vi phân của các hàm tương ứng với các chỉ số kích hoạt tại các điểm đã cho. Hơn nữa, về mặt ứng dụng, bài báo nghiên cứu sự tồn tại của chặn sai số toàn cục đối với các hệ bất đẳng thức tuyến tính và lồi vô hạn. Một số ví dụ cũng được cung cấp để giải thích những ưu điểm của kết quả được trình bày so với những kết quả hiện có trong các tài liệu tham chiếu gần đây.

\section{Bản nhận xét khoa học}

Bài báo này nghiên cứu sự tồn tại và đặc trưng của các chặn sai số toàn cục cho các hàm lồi trong không gian Euclid, cung cấp các điều kiện cần và đủ mà không cần áp đặt các yêu cầu bổ sung lên các hàm hoặc tập nghiệm của chúng. Bằng cách tập trung vào không gian Euclid, tác giả đưa ra các điều kiện dễ kiểm tra, nâng cao tính ứng dụng thực tế.

Hơn nữa, tác giả khám phá các tính chất của dưới vi phân của hàm chặn trên nhỏ nhất được hình thành từ một tập hợp tùy ý của các hàm lồi liên tục. Cụ thể, tác giả chứng minh rằng, dưới một giả định thích hợp, dưới vi phân của hàm chặn trên nhỏ nhất  này tương ứng với bao lồi của các dưới vi phân của các hàm thành phần tại các chỉ số đang hoạt động. Kết quả này mở rộng hiểu biết của chúng ta về mối quan hệ giữa các dưới vi phân của các hàm thành phần và hàm chặn trên nhỏ nhất của chúng, cung cấp một góc nhìn mới về phân tích lồi.

Các ứng dụng của những phát hiện lý thuyết này đặc biệt có liên quan đến việc nghiên cứu các chặn sai số toàn cục trong các hệ thống được đặc trưng bởi các bất đẳng thức tuyến tính và lồi vô hạn. Kết quả của tác giả không chỉ tổng quát hóa các lý thuyết hiện có mà còn cung cấp các tiêu chí thực tế dễ áp dụng hơn.

Xuyên suốt bài báo, nhiều ví dụ được đưa ra để minh họa những ưu điểm của các điều kiện đề xuất so với các điều kiện hiện có, chứng minh hiệu quả và tính hữu dụng cải thiện của phương pháp tiếp cận của chúng tôi trong nhiều ngữ cảnh khác nhau. Các ví dụ này nhấn mạnh tầm quan trọng thực tiễn của những đóng góp lý thuyết của tác giả, làm nổi bật tiềm năng ảnh hưởng đến việc giải quyết các vấn đề tối ưu hóa phức tạp liên quan đến các hàm lồi.