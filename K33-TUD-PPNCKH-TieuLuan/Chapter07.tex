\chapter{LỰA CHỌN ĐỀ TÀI NGHIÊN CỨU KHOA HỌC}

Trong chương này, chúng tôi thảo luận về vấn đề lựa chọn đề tài nghiên cứu khoa học. Đây là bước đầu tiên và cũng bước quan trọng để bắt đầu thực hiện một đề tài nghiên cứu khoa học.

\section{Xác định đề tài, nhiệm vụ và đối tượng nghiên cứu}

Bước này là bắt đầu của quá trình nghiên cứu, nơi mà nhà nghiên cứu xác định rõ ràng đề tài cụ thể mà họ sẽ nghiên cứu. Việc lựa chọn đề tài phải phù hợp với sở thích nghiên cứu của nhà nghiên cứu, đáp ứng nhu cầu hiện tại của cộng đồng khoa học và có khả năng đóng góp vào lĩnh vực nghiên cứu đang phát triển.

Xem xét ví dụ: Một nhóm nghiên cứu quyết định thực hiện nghiên cứu về "Tác động của việc sử dụng công nghệ AI trong giáo dục cơ sở và ảnh hưởng đến sự học tập của học sinh trung học." Đề tài đã được xác định rõ ràng là tác động của công nghệ AI trong giáo dục cơ sở. Nhiệm vụ của nghiên cứu là đo đạc và phân tích tác động của công nghệ AI đối với sự học tập của học sinh trung học. Đối tượng nghiên cứu là học sinh trung học, và cụ thể hơn, nhóm nghiên cứu có thể quyết định nghiên cứu trên một nhóm học sinh ở một số trường học cụ thể.

\section{Xác định mục tiêu nghiên cứu}

Sau khi xác định đề tài, mục tiêu nghiên cứu được xác định để chỉ ra những gì mà nghiên cứu sẽ cố gắng đạt được. Mục tiêu này cần phải rõ ràng, cụ thể và có thể đo lường được để hướng dẫn cho quá trình nghiên cứu.

Xem xét ví dụ: Mục tiêu của nghiên cứu là đánh giá cụ thể các ảnh hưởng của việc áp dụng công nghệ AI trong giáo dục, bao gồm cả mặt tích cực và tiêu cực, đến hiệu quả học tập của học sinh trung học. 

\section{Xác định mục đích nghiên cứu}

Mục đích nghiên cứu phản ánh ý nghĩa và giá trị thực tế của nghiên cứu đối với cộng đồng khoa học và xã hội. Nó giúp định hướng và làm rõ lý do tại sao nghiên cứu này cần phải được thực hiện và những lợi ích mà nó mang lại.

Xem xét ví dụ: Mục đích của nghiên cứu là cung cấp các dữ liệu và hiểu biết mới về tác động của công nghệ AI trong giáo dục cơ sở, từ đó đưa ra các đề xuất cải tiến và chính sách hỗ trợ để tối ưu hóa sự áp dụng công nghệ này trong học tập. 

\section{Đưa ra các câu hỏi nghiên cứu}

Các câu hỏi nghiên cứu được xây dựng dựa trên mục tiêu nghiên cứu để chỉ ra những khía cạnh cụ thể mà nghiên cứu sẽ điều tra và giải quyết. Các câu hỏi này giúp hướng dẫn cho việc thu thập dữ liệu và phân tích kết quả.

Ví dụ, ta có thể xây dựng các câu hỏi nghiên cứu có thể bao gồm:
\begin{itemize}
    \item Câu hỏi 1: Công nghệ AI làm thay đổi thế nào trong phương pháp giảng dạy và học tập so với các phương pháp truyền thống?
    \item Câu hỏi 2: Tác động của công nghệ AI đối với kết quả học tập và năng lực sáng tạo của học sinh như thế nào?
    \item Câu hỏi 3: Các yếu tố nào ảnh hưởng đến sự thành công của việc tích hợp công nghệ AI trong giáo dục cơ sở?
\end{itemize}
Các câu hỏi nghiên cứu phải phản ánh sự tò mò và nhu cầu giải quyết vấn đề trong lĩnh vực nghiên cứu. Chúng hướng dẫn cho việc thu thập dữ liệu và phân tích để giải quyết mục tiêu đã đặt ra.

\section{Đặt ra các giả thuyết ban đầu}

Giả thuyết ban đầu là các giả định dựa trên kiến thức hiện có và những quan sát sơ bộ, mà nghiên cứu sẽ kiểm tra và xác minh tính đúng đắn thông qua quá trình nghiên cứu.

Ví như đối về đề tài trên, ta có thể xây dựng các giả thuyết ban đầu:
\begin{itemize}
    \item Giả thuyết 1: Công nghệ AI sẽ cải thiện hiệu quả giảng dạy và học tập so với các phương pháp truyền thống.
    \item Giả thuyết 2: Học sinh sử dụng công nghệ AI có khả năng phát triển năng lực sáng tạo cao hơn.
    \item Giả thuyết 3: Điều kiện hạ tầng và sự chuẩn bị của giáo viên ảnh hưởng đến hiệu quả của công nghệ AI trong giảng dạy.
\end{itemize}

\section{Xác định đối tượng khảo sát và phạm vi nghiên cứu}

Đối tượng khảo sát và phạm vi nghiên cứu xác định các đơn vị nghiên cứu và khuôn khổ của nghiên cứu. Việc chọn đúng đối tượng và phạm vi nghiên cứu quyết định đến tính đại diện và áp dụng của kết quả nghiên cứu. 

Ví dụ: Đối tượng khảo sát là các học sinh trung học từ một số trường công và tư ở thành phố A. Phạm vi nghiên cứu bao gồm việc phân tích tác động của công nghệ AI trong giáo dục cơ sở trong vòng 2 năm.

Xác định rõ đối tượng và phạm vi nghiên cứu giúp đảm bảo tính đại diện và áp dụng của kết quả nghiên cứu, đồng thời hạn chế rủi ro và tối ưu hóa tài nguyên nghiên cứu.

\section{Vai trò của lựa chọn đề tài trong nghiên cứu khoa học}

Quá trình lựa chọn đề tài trong nghiên cứu khoa học không chỉ đơn thuần là một quyết định đơn giản mà là một chuỗi các bước chi tiết và logic. Việc thực hiện mỗi bước một cách cẩn thận và hợp lý sẽ đảm bảo rằng nghiên cứu có một nền tảng vững chắc và có khả năng đem lại những kết quả có giá trị. Đặc biệt, việc xác định mục đích, mục tiêu, và các câu hỏi nghiên cứu cần phải rõ ràng và cụ thể để giúp định hướng và hướng dẫn cho toàn bộ quá trình nghiên cứu. Những quyết định này cũng đóng vai trò quan trọng trong việc phát triển kiến thức và ứng dụng của nghiên cứu vào thực tiễn.
