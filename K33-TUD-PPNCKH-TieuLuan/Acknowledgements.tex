% \chapter*{LỜI CAM ĐOAN}
% \addcontentsline{toc}{chapter}{{\bf LỜI CAM ĐOAN}}
% Tôi cam đoan luận văn thạc sĩ ngành Toán ứng dụng, với đề tài \emph{Tổng quát hóa tính lồi trong không gian vô hạn chiều} là công trình khoa học do Tôi thực hiện dưới sự hướng dẫn của GS. TS. Nguyễn Văn B. 
% \\\\
% Những kết quả nghiên cứu của tiểu luận hoàn toàn trung thực và chính xác.
% \\\\\\
% \begin{flushright}
	
% 	\begin{tabular}{@{}c@{}}
% 		\textit{Học viên cao học}\\
% 		(Ký tên, ghi họ tên)\\\\\\\\ 
%         \textbf{Lê Nhựt Nam}
% 	\end{tabular}
	
% \end{flushright}
% \thispagestyle{empty}
\chapter*{LỜI CẢM ƠN}
\addcontentsline{toc}{chapter}{{\bf LỜI CẢM ƠN}}

Lời đầu tiên, tôi xin phép gửi lời cảm ơn chân thành đến các Thầy hướng dẫn của môn học Phương pháp nghiên cứu khoa học - GS. TS. Bùi Xuân Hải và PGS. TS Mai Hoàng Biên - giảng viên khoa Toán - Tin học, trường Đại học Khoa học Tự nhiên, Đại học Quốc gia TP. HCM đã trực tiếp hướng dẫn và giúp đỡ tận tình trong suốt quá trình nghiên cứu thực hiện tiểu luận này. Nhờ vào những định hướng, và góp ý quý giá của thầy, tôi đã hoàn thành trọn vẹn đề tài ltiểu luận của mình.

Tiếp theo, tôi xin gửi lời cảm ơn đến quý Thầy, Cô trong khoa Toán - Tin học, trường Đại học Khoa học Tự nhiên, Đại học Quốc gia TP. HCM đã nhiệt tình giảng dạy đã truyền đạt cho tôi những kiến thức sâu sắc về mặt chuyên môn lý thuyết và ứng dụng thực tiễn trong suốt quá trình học tập ở trường. Những điều này đã góp phần quan trọng trong việc hoàn thành tiểu luận này của tôi.

\begin{flushright}
		\begin{tabular}{@{}c@{}}
			\bfseries Lê Nhựt Nam
		\end{tabular}
\end{flushright}
\thispagestyle{empty}