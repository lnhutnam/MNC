\chapter{PHẨM CHẤT TRONG NGHIÊN CỨU KHOA HỌC}

Trong chương này, chúng tôi nghiên cứu về vấn đề các phẩm chất trong nghiên cứu khoa học. Chúng tôi dựa trên nhận định của GS. Ngô Bảo Châu – cựu học sinh của Khối THPT chuyên Toán, Trường ĐHKHTN, ĐHQGHN về vấn đề này. Một công trình khoa học cần có là 3 phẩm chất theo thứ tự: Đúng và trung thực, mới và hay. Nhưng quan trọng nhất là đúng và trung thực.

\section{Đúng trong nghiên cứu khoa học}

Trong nghiên cứu khoa học, tính đúng đắn là vô cùng quan trọng. Đúng đắn không chỉ đơn thuần là việc áp dụng các phương pháp và quy trình nghiên cứu một cách chính xác, mà còn liên quan đến sự khai thác hiệu quả những công cụ và kiến thức hiện đại nhất có sẵn. Các nhà nghiên cứu cần đảm bảo rằng mọi hoạt động nghiên cứu từ thu thập dữ liệu, thiết kế thí nghiệm, phân tích số liệu đến báo cáo kết quả đều được thực hiện với mức độ chính xác và khách quan cao nhất có thể.

Trong quá trình nghiên cứu, tính đúng đắn yêu cầu sự tỉ mỉ và cẩn trọng. Phải đảm bảo rằng mọi dữ liệu thu thập là chính xác và không bị sai sót, các phương pháp thí nghiệm được thiết kế và thực hiện đúng theo đúng qui trình khoa học. Điều này đảm bảo rằng kết quả nghiên cứu đưa ra là có tính xác thực và có thể tái sản xuất được bởi các nhà nghiên cứu khác.

\section{Trung thực trong nghiên cứu khoa học}

Trung thực là một yếu tố căn bản và không thể thiếu trong nghiên cứu khoa học. Nó đòi hỏi các nhà nghiên cứu phải có sự thẳng thắn và minh bạch trong tất cả các khâu của quá trình nghiên cứu, từ thu thập dữ liệu, xử lý số liệu, đến báo cáo kết quả. Trung thực bảo đảm rằng mọi thông tin được đưa ra là chân thực và không bị bóp méo hoặc làm sai lệch.

Nghiên cứu khoa học trung thực đòi hỏi sự trung thực với chính mình và với cộng đồng khoa học. Các tác giả cần phải báo cáo mọi phương pháp và quy trình nghiên cứu một cách minh bạch và chính xác, đồng thời phải công bố cả những kết quả không như mong đợi. Điều này giúp cảnh báo và ngăn ngừa sự phát sinh của bias (thiên lệch) trong nghiên cứu, giúp bảo vệ uy tín cá nhân và uy tín của tổ chức nghiên cứu.

\section{Mới và hay trong nghiên cứu khoa học}

Mới và hay là những tiêu chí quan trọng để đánh giá giá trị và ảnh hưởng của một công trình nghiên cứu trong cộng đồng khoa học. Mới đề cập đến tính sáng tạo và đổi mới của các ý tưởng nghiên cứu, tức là khả năng đưa ra các giải pháp hoặc phương pháp mới cho các vấn đề nghiên cứu hiện tại. Các nghiên cứu mới thường mang lại những thông tin, dữ liệu hoặc thậm chí các lý thuyết mới mà trước đó chưa từng được khám phá hoặc áp dụng, đóng góp một cách tích cực vào sự phát triển và tiến bộ của lĩnh vực đó.

Một nghiên cứu hay không chỉ đơn giản là mới mà còn phải đạt được sự gợi mở, thú vị và có giá trị sâu sắc trong lĩnh vực nghiên cứu tương ứng. Những nghiên cứu hay thường được đánh giá dựa trên khả năng giải quyết các vấn đề phức tạp, cung cấp các giải pháp hiệu quả, và góp phần vào sự phát triển và tiến bộ của lĩnh vực đó. Sự kết hợp giữa sự mới mẻ và sự hấp dẫn làm nên sự thành công của các nghiên cứu khoa học đáng chú ý, và cũng là tiêu chí để các nhà nghiên cứu và cộng đồng khoa học đánh giá và công nhận một công trình nghiên cứu.