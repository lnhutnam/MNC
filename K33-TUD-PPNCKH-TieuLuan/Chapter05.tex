\chapter{MỤC ĐÍCH VÀ MỤC TIÊU NGHIÊN CỨU}

Trong chương này, chúng tôi trình bày hai vấn đề cơ sở và bắt đầu trong nghiên cứu khoa học: mục đích và mục tiêu nghiên cứu. Chúng tôi tập trung phân tích sư khát biệt giữa hai khái niệm này.


\section{Thế nào là mục tiêu nghiên cứu?}

Mục tiêu nghiên cứu (research objective) là những nội dung cụ thể cần được xem xét và làm rõ trong khuôn khổ đối tượng nghiên cứu đã xác định, nhằm trả lời câu hỏi "Nghiên cứu cái gì?". Đây là các mục tiêu cụ thể mà nhà nghiên cứu đặt ra để đạt được trong quá trình nghiên cứu, giúp định hình phạm vi và hướng đi của nghiên cứu.

Vai trò của mục tiêu nghiên cứu như sau:
\begin{itemize}
    \item \textbf{Định Hướng Nghiên Cứu}: Mục tiêu nghiên cứu giúp xác định rõ ràng các vấn đề, hiện tượng hoặc mối quan hệ mà nghiên cứu sẽ tập trung vào. Điều này giúp nghiên cứu không bị lan man, tập trung vào những nội dung cần thiết và có giá trị.
    \item \textbf{Xác Định Phương Pháp Nghiên Cứu}: Dựa trên mục tiêu nghiên cứu, nhà nghiên cứu có thể lựa chọn các phương pháp và công cụ nghiên cứu phù hợp để thu thập và phân tích dữ liệu. Ví dụ, nếu mục tiêu là xác định mối quan hệ giữa hai biến số, thì phương pháp nghiên cứu định lượng có thể được áp dụng.
    \item \textbf{Xây Dựng Câu Hỏi Nghiên Cứu}: Từ mục tiêu nghiên cứu, các câu hỏi nghiên cứu được xây dựng để làm rõ từng khía cạnh cụ thể của vấn đề cần nghiên cứu. Các câu hỏi này giúp hướng dẫn quá trình thu thập dữ liệu và phân tích, đảm bảo rằng mọi khía cạnh của mục tiêu đều được xem xét.
\end{itemize}

Quá trình xây dựng mục tiêu nghiên cứu như sau:
\begin{itemize}
    \item \textbf{Xác Định Vấn Đề Nghiên Cứu}: Trước tiên, nhà nghiên cứu phải xác định được vấn đề hoặc hiện tượng cần nghiên cứu. Điều này có thể xuất phát từ một khoảng trống trong kiến thức hiện tại hoặc từ một nhu cầu thực tiễn cần được giải quyết.
    \item \textbf{Thiết Lập Mục Tiêu Tổng Quát}: Mục tiêu tổng quát thường là một tuyên bố rộng, phản ánh mục đích chính của nghiên cứu. Nó thường liên quan đến việc khám phá, mô tả hoặc giải thích một hiện tượng cụ thể.
    \item \textbf{Thiết Lập Mục Tiêu Cụ Thể}: Mục tiêu cụ thể là những tuyên bố chi tiết hơn, xác định các khía cạnh cụ thể của vấn đề cần được xem xét. Các mục tiêu cụ thể này thường liên quan trực tiếp đến các câu hỏi nghiên cứu.
\end{itemize}

Xem xét ví dụ như sau: Nếu nghiên cứu tập trung vào việc "Đánh giá ảnh hưởng của việc học trực tuyến đến kết quả học tập của sinh viên đại học", mục tiêu nghiên cứu có thể được thiết lập như sau:
\begin{itemize}
    \item \textbf{Mục Tiêu Tổng Quát}: Đánh giá tổng thể ảnh hưởng của việc học trực tuyến đến kết quả học tập của sinh viên.
    \item \textbf{Mục Tiêu Cụ Thể}: So sánh kết quả học tập giữa sinh viên học trực tuyến và sinh viên học truyền thống; Xác định các yếu tố ảnh hưởng đến hiệu quả học tập trực tuyến của sinh viên; Khảo sát mức độ hài lòng của sinh viên đối với hình thức học trực tuyến.
\end{itemize}

Nói tóm lại, mục tiêu nghiên cứu là một phần quan trọng trong quá trình nghiên cứu khoa học, giúp định hướng và tập trung vào các nội dung cần thiết. Việc xác định mục tiêu rõ ràng và chi tiết không chỉ giúp nghiên cứu đạt được kết quả chính xác mà còn góp phần vào việc phát triển kiến thức mới và ứng dụng vào thực tiễn.

\section{Thế nào là mục đích nghiên cứu?}


Mục đích nghiên cứu (research purpose) là ý nghĩa thực tiễn của nghiên cứu, trả lời cho câu hỏi "Nghiên cứu nhằm vào việc gì?" hoặc "Nghiên cứu để phục vụ cho cái gì?". Đây là yếu tố quan trọng xác định lý do và giá trị của việc tiến hành một nghiên cứu cụ thể, đồng thời định hình hướng đi và mục tiêu cuối cùng của nghiên cứu.

Mục đích nghiên cứu cần phải đảm bảo các ý sau:
\begin{itemize}
    \item \textbf{Ý Nghĩa Thực Tiễn}: Mục đích nghiên cứu nhấn mạnh tầm quan trọng và ứng dụng thực tế của nghiên cứu trong đời sống, khoa học hoặc công nghệ. Nó giúp xác định rõ ràng lý do vì sao nghiên cứu này cần được thực hiện và những lợi ích cụ thể mà nó mang lại.
    \item \textbf{Định Hướng và Phục Vụ}: Mục đích nghiên cứu giúp định hướng cho toàn bộ quá trình nghiên cứu, từ việc xác định vấn đề, xây dựng mục tiêu và câu hỏi nghiên cứu, đến lựa chọn phương pháp và phân tích kết quả. Nó trả lời cho câu hỏi nghiên cứu nhằm phục vụ cho điều gì, chẳng hạn như giải quyết một vấn đề xã hội, cải tiến một quy trình công nghệ, hay mở rộng kiến thức trong một lĩnh vực khoa học cụ thể.
    \item \textbf{Động Lực và Lý Do}: Mục đích nghiên cứu cung cấp động lực và lý do cho các nhà nghiên cứu và các bên liên quan, giúp họ hiểu rõ tầm quan trọng của nghiên cứu và cam kết với quá trình thực hiện. Nó làm rõ những lợi ích tiềm năng mà nghiên cứu có thể mang lại, không chỉ cho lĩnh vực nghiên cứu cụ thể mà còn cho cộng đồng hoặc xã hội nói chung.
\end{itemize}

Xem xét ví dụ: Nếu nghiên cứu tập trung vào "Tìm hiểu tác động của ô nhiễm không khí đến sức khỏe cộng đồng tại thành phố X", mục đích nghiên cứu có thể được trình bày như sau:
\begin{itemize}
    \item \textbf{Giải Quyết Vấn Đề Thực Tiễn}: Nghiên cứu nhằm xác định mức độ và loại ô nhiễm không khí đang ảnh hưởng đến sức khỏe của người dân thành phố X, từ đó đưa ra các giải pháp giảm thiểu tác động tiêu cực.
    \item \textbf{Cung Cấp Thông Tin Cho Chính Sách}: Kết quả nghiên cứu sẽ cung cấp dữ liệu quan trọng để hỗ trợ các cơ quan chức năng trong việc xây dựng và triển khai các chính sách bảo vệ môi trường và sức khỏe cộng đồng.
    \item \textbf{Nâng Cao Nhận Thức Cộng Đồng}: Mục đích nghiên cứu còn nhằm nâng cao nhận thức của người dân về vấn đề ô nhiễm không khí và khuyến khích họ tham gia vào các hoạt động bảo vệ môi trường.
\end{itemize}

Nói tóm lại, mục đích nghiên cứu là một thành phần cốt lõi trong quá trình nghiên cứu, xác định rõ ý nghĩa thực tiễn và giá trị của nghiên cứu. Nó không chỉ định hướng và định hình toàn bộ quá trình nghiên cứu mà còn cung cấp lý do và động lực cho các nhà nghiên cứu và các bên liên quan. Việc xác định mục đích nghiên cứu rõ ràng và chi tiết đảm bảo rằng nghiên cứu không chỉ mang lại kiến thức mới mà còn có thể ứng dụng và mang lại lợi ích cụ thể cho xã hội.

\section{Sự khác biệt giữa mục đích và mục tiêu nghiên cứu}

Mặc dù mục đích và mục tiêu nghiên cứu đều là những thành phần quan trọng trong quá trình nghiên cứu, nhưng chúng khác nhau về bản chất và vai trò. Mục đích nghiên cứu (research purpose) là ý nghĩa thực tiễn của nghiên cứu, trả lời cho câu hỏi "Nghiên cứu nhằm vào việc gì?" hoặc "Nghiên cứu để phục vụ cho cái gì?". Nó định rõ lý do và giá trị thực tế của việc thực hiện nghiên cứu, nhấn mạnh tầm quan trọng và ứng dụng của nghiên cứu trong đời sống hoặc khoa học. Mục đích nghiên cứu cung cấp động lực và lý do cho các nhà nghiên cứu và các bên liên quan, giúp họ hiểu rõ tầm quan trọng của nghiên cứu và cam kết với quá trình thực hiện.

Ngược lại, mục tiêu nghiên cứu (research objective) là những nội dung cụ thể cần được xem xét và làm rõ trong khuôn khổ đối tượng nghiên cứu đã xác định, nhằm trả lời câu hỏi "Nghiên cứu cái gì?". Mục tiêu nghiên cứu xác định rõ ràng các vấn đề, hiện tượng hoặc mối quan hệ mà nghiên cứu hướng đến, đóng vai trò làm kim chỉ nam cho toàn bộ quá trình nghiên cứu. Nó giúp xác định phạm vi, hướng đi và phương pháp nghiên cứu, từ đó xây dựng các câu hỏi nghiên cứu cụ thể để điều tra và phân tích.

Tóm lại, mục đích nghiên cứu nhấn mạnh vào lý do và ý nghĩa thực tiễn của nghiên cứu, còn mục tiêu nghiên cứu tập trung vào các nội dung cụ thể và chi tiết cần được nghiên cứu để đạt được mục đích đó. Mục đích nghiên cứu định hướng tổng thể và giá trị của nghiên cứu, trong khi mục tiêu nghiên cứu xác định các bước và phương pháp cụ thể để thực hiện nghiên cứu.