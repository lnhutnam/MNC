\chapter{PHÂN TÍCH VỀ TRI THỨC KINH NGHIỆM VÀ TRI THỨC KHOA HỌC}

Trong chương này, chúng tôi trình bày và thảo luận về vấn đề tri thức. Một cách cụ thể, chúng tôi tập trung phân tích sự khác biệt giữa tri thức kinh nghiệm và tri thức khoa học.

\section{Thế nào là tri thức?}

Ở bất kỳ thời đại nào, con người luôn khao khát chạm tới sự hiểu biết và vươn đến cái tận cùng của vạn vật. Cuộc sống con người là hữu hạn, từ thời gian đến khả năng của bản thân. Thế nhưng trong sự khó khăn và giới hạn đó, vẫn tồn tại "vô hạn", đó chính là sự khát khao vô cùng không ngơi nghỉ để \emph{tìm kiếm chân lý}, tìm kiếm con đường chạm đến \emph{căn nguyên} của bản thân mình. Vấn đề tri thức là một trong những vấn đề đầu tiên mà con người đặt ra trên con đường đó. Trong giới hạn của tiểu luận này, chúng tôi trình bày vấn đề này xoay quanh ba tư tưởng ứng với ba nhà triết gia nổi tiếng trong thời kỳ đầu Hy Lạp cổ đại.

Trước hết, có lẽ phải nhắc đến Socrates. Ông là nhà triết gia đầu tiên thảo luận đến những vấn đề bản chất con người. Ông cho rằng để đạt được bản chất cao nhất của con người thì họ phải hướng tới sự hiện đích thực mà đạo đức là con đường quan trọng nhất. Ông đồng hóa hai khái niệm lớn là tri thức và nhân đức. Một câu nói nổi tiếng của ông chính là "tôi chỉ biết một điều là tôi không biết gì hết" đã gợi lên những suy ngẫm về vấn đề tri thức và bản chất thực tại của thế giới. Điều này có thể thấy được "nhân đức" trong con người ông, một sự khiêm nhường đối với kho tàn tri thức của nhân loại.

Từ những vấn đề của Socrates, học trò của ông, Plato đã đưa tư tưởng của người thầy mình lên một tầm cao mới. Ông cũng khẳng định rằng tri thức là nhân đức, và hơn nữa ông cũng đưa ra phương pháp có thể dẫn con người đạt được đến chân lý, hiểu biết đích thực và con đường đấy cũng đưa con người đến hạnh phúc thật sự. Theo ông, con người cần được khai sáng, được dẫn lối và đi tới sự hiểu biết về thực tại cuộc sống để thoát khỏi những u mê của ngu dốt. Nói cách khác, con người cần hoán cải nhờ tri thức, tức là khi cái biết đích thực xảy ra với con người tại thời điểm mà họ thấu hiểu được thực tại và thế họ sẽ không thể làm ngược lại với những hiểu biết của họ được. 

Còn đối với Aristotle , việc nhận thức vấn đề và tri thức cần lời giải thích lý do hiện hữu của nó. Theo ông, tri thức được đề cập trên bốn nguyên nhân: nguyên nhân chất thể, mô thể, tác động và nguyên nhân cùng đích. Và thông ông, con người có thể đạt được tri thức nhờ quá trình chiêm nghiệm. Khi có một cuộc sống chiêm nghiệm, con người có thể đạt được sự nhân thức đích thực về bốn nguyên nhân vừa nêu. Và khi đã đạt được tri thức đích thực của sự vật, con người sẽ chọn lựa lối sống để đạt được tới hạnh phúc. 

Như vậy, thông qua những quan niệm về tri thức của các nhà triết gia, ta có thể hiểu tri thức là một sự hiểu biết được tích lũy thông qua một quá trình nào đó. Và cái đích cuối cùng của nó là hướng đến cái thiện, cuộc sống hạnh phúc.

\section{Thế nào là khoa học?}

\epigraph{"Science is a set of rules that keep the scientists from lying to each other."\\
\underline{Tạm dịch}: Khoa học là một tập hợp các quy tắc để mà giữ các nhà khoa học không nói dối lẫn nhau.
}{\textit{Kenneth S. Norris, được trích dẫn trong False Prophets (1988) bởi Alexander Kohn.}}

Khoa học là một hệ thống tri thức của con người về các sự vật, hiện tượng, và các quy luật hoạt động của vật chất, trong tự nhiên, xã hội và tư duy. Theo Wikipedia, khoa học không chỉ là một tập hợp các thông tin đơn lẻ mà là một hệ thống phức tạp được tổ chức và phát triển qua nhiều giai đoạn lịch sử, phản ánh nhận thức của con người về thế giới xung quanh.

Quá trình hình thành khoa học có thể được biểu diễn thành các bước như sau:
\begin{itemize}
    \item Quan Sát: Bắt đầu từ việc quan sát các hiện tượng tự nhiên và xã hội.
    \item Mô Tả: Tiếp theo là mô tả chi tiết những gì đã được quan sát.
    \item Đo Đạc: Đo đạc các thông số liên quan để có số liệu cụ thể.
    \item Thực Nghiệm: Thực hiện các thí nghiệm để kiểm tra và xác minh các giả thuyết.
    \item Phát Triển Lý Thuyết: Dựa trên kết quả thực nghiệm, phát triển các lý thuyết để mô tả và giải thích các hiện tượng.
    \item Tích Lũy và Tổ Chức Thông Tin: Quá trình thu thập, sắp xếp và tổ chức thông tin để hình thành tri thức khoa học.
\end{itemize}

Từ đó, con người hình thành và đạt được tri thức khoa học. Tri thức trong khoa học chính là toàn bộ lượng thông tin tích lũy qua các giai đoạn quan sát, thực nghiệm và lý thuyết hóa. Đây là kết quả của quá trình tích lũy kiến thức có hệ thống, giúp con người hiểu rõ hơn về cách mà vạn vật vận hành.

Một trong những động lực chính của khoa học là sự tò mò và hoài nghi. Tuy nhiên, thái độ hoài nghi không phải là bản năng tự nhiên của con người. Mọi người thường có xu hướng tin vào những gì họ được nghe. Nhưng để làm khoa học và đạt được tri thức khoa học, con người cần phải có thái độ hoài nghi, không dễ dàng tin vào những gì chưa được chứng minh. Trong nhiều trường hợp, những thông tin không đúng sự thật có thể được lặp lại nhiều lần và dần dần được chấp nhận như là sự thật. Nếu không có thái độ hoài nghi và không tự mình tìm hiểu, con người sẽ khó mà đạt được tri thức thực sự trong khoa học.

Quá trình nhận thức của con người được thực hiện ở nhiều trình độ và bằng nhiều phương thức khác nhau, tạo ra hai hệ thống tri thức chính:
\begin{itemize}
    \item Tri Thức Kinh Nghiệm: Được hình thành từ quá trình tích lũy kinh nghiệm thực tiễn và truyền lại qua các thế hệ.
    \item Tri Thức Khoa Học: Được phát triển dựa trên các phương pháp khoa học, mang tính hệ thống và có cơ sở lý thuyết chặt chẽ.
\end{itemize}

Nói tóm lại, khoa học là một hệ thống tri thức phức tạp và không ngừng phát triển, giúp con người hiểu biết sâu hơn về thế giới xung quanh. Để đạt được tri thức khoa học, con người cần có sự tò mò, thái độ hoài nghi và phương pháp khoa học. Tri thức khoa học không chỉ dừng lại ở việc thu thập thông tin mà còn phải tổ chức, sắp xếp và lý thuyết hóa để hiểu rõ hơn về bản chất của các hiện tượng.

\section{Thế nào là tri thức kinh nghiệm?}

Tri thức kinh nghiệm là hệ thống tri thức được tích lũy một cách ngẫu nhiên trong quá trình hoạt động của con người. Hệ thống này được hình thành qua các hoạt động thực tiễn như lao động sản xuất, chiến đấu bảo vệ cuộc sống của cộng đồng, thông qua hoạt động tư duy của con người và được truyền lại qua các thế hệ bằng nhiều con đường khác nhau.

Con người có thể hình dung được sự vật, biết cách phản ứng trước tự nhiên và biết cách ứng xử trong các quan hệ xã hội nhờ vào tri thức kinh nghiệm. Hệ thống tri thức này ngày càng phong phú và được tích lũy qua các năm, các thế hệ, đóng vai trò quan trọng trong sự phát triển của loài người.

Tri thức kinh nghiệm giúp con người nhận thức được những đặc điểm bên ngoài của sự vật, hiện tượng. Đây là cơ sở quan trọng cho sự hình thành tri thức khoa học. Tuy nhiên, tri thức kinh nghiệm thường mang tính rời rạc, thiếu cơ sở khoa học và bị hạn chế trong việc giải thích và vận dụng để cải tạo thế giới. Do đó, tri thức kinh nghiệm chỉ giúp con người phát triển đến một khuôn khổ nhất định mà chưa thể đi sâu vào bản chất các sự vật và hiện tượng. Tri thức kinh nghiệm có vai trò như sau:
\begin{itemize}
    \item \textbf{Nhận Thức và Phản Ứng}: Tri thức kinh nghiệm giúp con người hình dung và nhận thức được sự vật, biết cách phản ứng trước các hiện tượng tự nhiên và xã hội.
    \item \textbf{Ứng Xử Xã Hội}: Hệ thống tri thức này cung cấp cho con người những phương pháp ứng xử phù hợp trong các mối quan hệ xã hội.
    \item \textbf{Phát Triển Qua Thế Hệ}: Tri thức kinh nghiệm được tích lũy và truyền lại qua các thế hệ, làm phong phú thêm kho tàng kiến thức của nhân loại.
    \item \textbf{Cơ Sở Hình Thành Tri Thức Khoa Học}: Tri thức kinh nghiệm là nền tảng để phát triển tri thức khoa học, giúp con người nhận thức ban đầu về các đặc điểm bên ngoài của sự vật và hiện tượng.
\end{itemize}

Tuy nhiên, tri thức kinh nghiệm vẫn có những hạn chế nhất định
\begin{itemize}
    \item \textbf{Tính Rời Rạc}: Tri thức kinh nghiệm thường không được hệ thống hóa một cách chặt chẽ, mà mang tính rời rạc và phân tán.
    \item \textbf{Thiếu Cơ Sở Khoa Học}: Do không dựa trên cơ sở khoa học vững chắc, tri thức kinh nghiệm bị hạn chế trong việc giải thích các hiện tượng phức tạp và không thể đi sâu vào bản chất của sự vật.
    \item \textbf{Giới Hạn Trong Vận Dụng}: Khả năng vận dụng tri thức kinh nghiệm để cải tạo thế giới còn hạn chế, chỉ giúp con người phát triển đến một mức độ nhất định mà không thể tiếp tục khám phá và hiểu biết sâu hơn về tự nhiên và xã hội.
\end{itemize}

\section{Thế nào là tri thức khoa học?}

Tri thức khoa học là hệ thống tri thức được tích lũy một cách có hệ thống thông qua hoạt động nghiên cứu khoa học. Quá trình này được tiến hành theo một mục tiêu xác định và sử dụng các phương pháp khoa học để thu thập, phân tích và khái quát hóa thông tin. Khác với tri thức kinh nghiệm, tri thức khoa học không chỉ dừng lại ở việc thu thập những tập hợp số liệu và sự kiện ngẫu nhiên, mà còn khái quát hóa chúng thành cơ sở lý thuyết để hiểu rõ bản chất của sự vật và hiện tượng trong thế giới khách quan.

Tri thức khoa học được phát triển qua nhiều bước cụ thể:
\begin{itemize}
    \item \textbf{Quan sát và Thu Thập Dữ Liệu}: Bắt đầu từ việc quan sát các hiện tượng tự nhiên và xã hội, ghi chép lại các sự kiện và số liệu.
    \item \textbf{Thực Nghiệm và Phân Tích}: Tiếp theo là thực hiện các thí nghiệm để kiểm tra giả thuyết, sử dụng các phương pháp phân tích dữ liệu để xác định các mô hình và xu hướng.
    \item \textbf{Khái Quát Hóa}: Từ những dữ liệu thu thập được, các nhà khoa học phát triển các lý thuyết và mô hình để giải thích bản chất của các hiện tượng, khái quát hóa từ những trường hợp cụ thể để tạo ra các nguyên lý chung.
\end{itemize}

Tri thức khoa học đóng vai trò vô cùng quan trọng trong đời sống con người và sự phát triển của xã hội:
\begin{itemize}
    \item \textbf{Nhận Thức Thế Giới}: Tri thức khoa học giúp con người hiểu rõ hơn về thế giới xung quanh, từ các nguyên lý cơ bản của tự nhiên đến các hiện tượng xã hội phức tạp.
    \item \textbf{Cải Tạo Thế Giới}: Tri thức khoa học cung cấp nền tảng cho việc phát triển các công nghệ mới, cải tiến quy trình sản xuất, và giải quyết các vấn đề thực tiễn trong cuộc sống.
    \item \textbf{Thể Hiện trong Các Môn Học}: Các môn học như vật lý học, hóa học, sinh học, xã hội học, đều dựa trên tri thức khoa học để nghiên cứu và giảng dạy, giúp lan tỏa tri thức này đến các thế hệ sau.
\end{itemize}

Tri thức khoa học không chỉ giúp con người tìm hiểu và giải thích các hiện tượng mà còn cho phép kiểm nghiệm và đánh giá tính đúng đắn của các tri thức kinh nghiệm:
\begin{itemize}
    \item \textbf{Kiểm Nghiệm và Đánh Giá}: Các lý thuyết khoa học được kiểm nghiệm qua nhiều thí nghiệm và quan sát, đảm bảo tính chính xác và tin cậy. Điều này giúp loại bỏ những sai lầm và ngộ nhận từ tri thức kinh nghiệm.
    \item \textbf{Vận Dụng Sáng Tạo}: Tri thức khoa học có thể được áp dụng một cách sáng tạo vào nhiều lĩnh vực khác nhau, từ y học, kỹ thuật, đến kinh tế và xã hội, mở rộng khả năng giải quyết vấn đề và tạo ra những tiến bộ vượt bậc.
\end{itemize}

Tri thức khoa học góp phần quan trọng vào mọi thành tựu và tiến bộ trong lịch sử phát triển của văn minh nhân loại. Nó giúp con người nhận thức đúng đắn, đầy đủ về sự vật và hiện tượng, đồng thời đảm bảo sự phát triển bền vững:
\begin{itemize}
    \item \textbf{Thành Tựu Khoa Học và Công Nghệ}: Những phát minh và phát kiến khoa học đã thay đổi hoàn toàn cách con người sống và làm việc, từ việc khám phá ra điện, phát triển máy tính, đến công nghệ sinh học và y học hiện đại.
    \item \textbf{Phát Triển Kinh Tế và Xã Hội}: Tri thức khoa học thúc đẩy sự phát triển kinh tế qua việc cải tiến quy trình sản xuất, nâng cao năng suất và chất lượng sản phẩm. Đồng thời, nó cũng góp phần giải quyết các vấn đề xã hội như sức khỏe, giáo dục, và môi trường.
    \item C\textbf{ơ Sở Cải Tạo Thế Giới}: Tri thức khoa học cung cấp cơ sở lý thuyết và công cụ thực tiễn để con người có thể cải tạo thế giới, từ việc xây dựng các công trình kiến trúc, phát triển hạ tầng giao thông, đến việc bảo vệ và khôi phục môi trường tự nhiên.
\end{itemize}

Tóm lại, tri thức khoa học là một tài sản vô giá của loài người, không chỉ giúp chúng ta hiểu rõ hơn về thế giới mà còn cung cấp những công cụ cần thiết để cải thiện và phát triển xã hội. Sự phát triển của tri thức khoa học là động lực quan trọng cho mọi tiến bộ và thành tựu của nhân loại.

\section{Sự khác biệt giữa tri thức kinh nghiệm và tri thức khoa học}

Tri thức kinh nghiệm và tri thức khoa học là hai hệ thống tri thức của con người, khác nhau về nguồn gốc, cách thức hình thành và mức độ chính xác. Tri thức kinh nghiệm được tích lũy một cách ngẫu nhiên thông qua các hoạt động thực tiễn, lao động sản xuất, và qua quá trình tư duy của con người. Nó phản ánh những hiểu biết và kỹ năng được truyền lại qua các thế hệ, giúp con người hình dung về sự vật, phản ứng trước tự nhiên và ứng xử trong xã hội. Tri thức kinh nghiệm thường mang tính rời rạc, thiếu cơ sở khoa học và bị hạn chế trong việc giải thích các hiện tượng phức tạp, do đó chỉ giúp con người phát triển đến một mức độ nhất định.

Ngược lại, tri thức khoa học là hệ thống tri thức được tích lũy một cách hệ thống thông qua hoạt động nghiên cứu khoa học, với mục tiêu xác định và bằng những phương pháp khoa học. Tri thức khoa học tổng kết những tập hợp số liệu và sự kiện ngẫu nhiên để khái quát hóa thành các cơ sở lý thuyết về bản chất của sự vật và hiện tượng trong thế giới khách quan. Đây là kho tàng tri thức quan trọng của loài người, đóng vai trò thiết yếu trong việc nhận thức và cải tạo thế giới. Tri thức khoa học giúp con người hiểu rõ bản chất của các hiện tượng, kiểm nghiệm và đánh giá tính đúng đắn của tri thức kinh nghiệm, và vận dụng sáng tạo để giải quyết các vấn đề trong nhiều lĩnh vực khác nhau.

Tóm lại, tri thức kinh nghiệm và tri thức khoa học đều đóng góp quan trọng vào sự phát triển của nhân loại, nhưng tri thức khoa học vượt trội hơn ở tính hệ thống, cơ sở lý thuyết và khả năng giải thích, cải tạo thế giới một cách toàn diện và chính xác hơn.

