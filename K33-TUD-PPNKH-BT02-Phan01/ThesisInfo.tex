\chapter*{TRANG THÔNG TIN LUẬN VĂN}
\addcontentsline{toc}{chapter}{{\bf TRANG THÔNG TIN LUẬN VĂN}}

\begin{flushleft}
Tên đề tài luận văn: Thực hành thiết kế luận văn bằng \LaTeX

Ngành: Toán ứng dụng

Mã số ngành: 84 60 112

Họ tên học viên cao học: Lê Nhựt Nam

Khóa đào tạo: 33/2022

Người hướng dẫn khoa học: TS. Trịnh Thanh Đèo

Cơ sở đào tạo: Trường Đại học Khoa học Tự nhiên, ĐHQG.HCM 
\end{flushleft}

\section*{1. TÓM TẮT NỘI DUNG LUẬN VĂN}
\lipsum[1-2]

\section*{2. NHỮNG KẾT QUẢ MỚI CỦA LUẬN VĂN}
\lipsum[1-2]
% Cần nêu lên các ý chính như sau:
% \begin{itemize}
%     \item Những kiến nghị, nhận định, luận điểm, kết quả cụ thể của riêng tác giả rút ra được sau khi hoàn thành đề tài luận văn
%     \item Những ý kiến, nhận định, luận điểm, kết quả này phải là mới, chưa được những người nghiên cứu trước nêu ra. Không nêu lại những ý kiến nhận định, luận điểm, kết quả có tính chất giáo khoa, kinh điển hay đã biết, lặp lại của người khác
%     \item Những kết luận mới này cần nêu rất cụ thể, ngắn gọn, lượng hóa được và cần được diễn đạt một cách khách quan, khoa học có thể chuyên sâu. Không dùng cụm từ mang tính chất đánh gia như “lần đầu tiên”, “đầy đủ nhất”, “sâu sắc nhất”, “rất quan trọng” hay những từ quá chung chung có thể đúng cho bất kỳ luận văn nào
%     \item Không mô tả hay nêu lại những công việc mà tác giả đã tiến hành trong quá trình thực hiện đề tài như: “đã xây dựng”, “đã hoàn thiện”, “đã nêu lên”, “đã làm sáng tỏ”, “đã nghiên cứu một cách có hệ thống” hay “đã tổng kết, hệ thống hóa”
% \end{itemize}

\section*{3. CÁC ỨNG DỤNG/ KHẢ NĂNG ỨNG DỤNG TRONG THỰC TIỄN HAY NHỮNG VẤN ĐỀ CÒN BỎ NGỎ CẦN TIẾP TỤC NGHIÊN CỨU}
\lipsum[1-1]

\vspace{4\baselineskip}
\begin{table}[H]
\begin{adjustbox}{max width =\textwidth}
\begin{tabular}{p{8.44cm}p{8.4cm}}
\multicolumn{1}{p{8.44cm}}{
\centering \textbf{TẬP THỂ CÁN BỘ HƯỚNG DẪN} \newline
\centering
(Ký tên, họ tên) \newline
} &
\multicolumn{1}{p{8.4cm}}{
\centering \textbf{HỌC VIÊN CAO HỌC} \newline
\centering
(Ký tên, họ tên) \newline
} \\
\end{tabular}
\end{adjustbox}
\end{table}
\vspace{2\baselineskip}
\begin{center}
    \textbf{XÁC NHẬN CỦA CƠ SỞ ĐÀO TẠO}
\end{center}
\begin{center}
    \textbf{HIỆU TRƯỞNG}
\end{center}